\newglossaryentry{api}
{
    name=API,
    description={Acronym for Application Programming Interface, a system that lets two applications communicate}
}

\newglossaryentry{on-premises}
{
    name=on-premises,
    description={A setup where software runs on the own machines of a user or organization, instead of cloud providers}
}

\newglossaryentry{ml}
{
    name=ML,
    description={Acronym for machine learning, which is a branch of computer science}
}

\newglossaryentry{cri}
{
    name=CRI,
    description={Acronym for Container Runtime Interface, an unified interface for container runtimes}
}

\newglossaryentry{load-balancer}
{
    name=load balancer,
    description={Distributes requests between given machines. This is used to scale the capacity of a system}
}

\newglossaryentry{ssl}
{
    name=SSL,
    description={Acronym for Secure Sockets Layer, or more properly, TLS (Transport Layer Security). A set of technologies allowing channels to be encrypted, preventing man-in-the-middle attacks}
}

\newglossaryentry{tuple}
{
    name=tuple,
    description={A finite ordered list. A tuple of n elements is called n-tuple}
}

\newglossaryentry{risc}
{
    name=RISC,
    description={Acronym for Reduced Instruction Set Computing which refers to an instruction set architecture that uses relatively few instructions}
}

\newglossaryentry{read-only}
{
    name=read-only,
    description={A register or memory location that can't be written to, only read from}
}

\newglossaryentry{fp}
{
    name=FP,
    description={Acronym for frame pointer. A highly specialized type of CPU register used for managing function calls}
}

\newglossaryentry{pc}
{
    name=PC,
    description={Acronym for program counter. A specialized CPU register that points to the next instruction to be executed}
}

\newglossaryentry{stack}
{
    name=stack,
    description={A CPU register that keeps track of the call stack}
}

\newglossaryentry{verifier}
{
    name=verifier,
    description={The EBPF verified is used to inspect the code before loading it into the kernel, making sure it doesn't contain unsafe code that can harm the system}
}

\newglossaryentry{kprobe}
{
    name=kprobe,
    description={Hooks that allow to run EBPF code when specific kernel functions are called}
}

\newglossaryentry{uprobe}
{
    name=uprobe,
    description={Hooks that allow to run EBPF code when specific user space functions are called}
}

\newglossaryentry{socket-buffer}
{
    name=socket buffer,
    description={A structure to hold data from a packet. It enables easy access and modification from code}
}

\newglossaryentry{tc}
{
    name=TC,
    description={Acronym for Traffic Control, an user space program used for configuring the Linux packet scheduler}
}

\newglossaryentry{edt}
{
    name=EDT,
    description={Acronym for Earliest Departure Time, an algorithm to limit the bandwidth of a given data stream by delaying packets}
}

\newglossaryentry{cpu}
{
    name=CPU,
    description={Acronym for Central Processing Unit, referring to the processor of a machine}
}

\newglossaryentry{ram}
{
    name=RAM,
    description={Acronym for Random Access Memory, used to store all temporary data in a machine}
}

\newglossaryentry{tree}
{
    name=tree,
    description={A hierarchical data structure made up from nodes ordered in layers. Each node can have multiple children but only one parent}
}

\newglossaryentry{ingress}
{
    name=ingress,
    description={An network interface where transmitted data enters a machine or process}
}

\newglossaryentry{egress}
{
    name=egress,
    description={A network interface where data exits a machine or process and gets transmitted}
}

\newglossaryentry{tbf}
{
    name=TBF,
    description={Acronym for Token Bucket Filter, which is an algorithm to limit bandwidth of a data stream. This is achieved by dropping non-conformant packets}
}

\newglossaryentry{latency}
{
    name=latency,
    description={The delay between two hosts on a network}
}

\newglossaryentry{tcp}
{
    name=TCP,
    description={Acronym for Transmission Control Protocol, which is a reliable protocol for data transmission, but slower than UDP}
}

\newglossaryentry{udp}
{
    name=UDP,
    description={Acronym for User Datagram Protocol, an unreliable but fast protocol for data exchange}
}

\newglossaryentry{netcode}
{
    name=netcode,
    description={A part of an application that deals with the data exchange over a network. It has to ensure data is reliably and correctly transmitted}
}

\newglossaryentry{transport-layer}
{
    name=transport layer,
    description={A conceptual layer in the network stack of the OSI model. Protocols in this layer deal with host-to-host communication and data multiplexing between processes}
}

\newglossaryentry{compiled}
{
    name=compiled,
    description={A type of programming language that requires the source code to be translated to machine-readable code before execution}
}

\newglossaryentry{statically-typed}
{
    name=statically typed,
    description={A type of language where the type of a variable is known at compile-time}
}

\newglossaryentry{garbage-collector}
{
    name=garbage collector,
    description={A programming language feature that automatically cleans up unused variables on the heap}
}

\newglossaryentry{memory-safe}
{
    name=memory-safe,
    description={A type of language with safety measures that prevent various memory-related errors such as buffer overflows, out-of-bounds writes and the like}
}

\newglossaryentry{oop}
{
    name=OOP,
    description={Acronym for Object Oriented Programming, a popular paradigm built around the concept of data hiding - also called encapsulation}
}

\newglossaryentry{ide}
{
    name=IDE,
    description={Acronym for Integrated Development Environment, software with built-in features to support development}
}

\newglossaryentry{float}
{
    name=floating-point numbers,
    description={A rational number stored as exponent and mantissa}
}

\newglossaryentry{rng}
{
    name=PRNG,
    description={Acronym for pseudo random number generator, an algorithm that outputs numbers that are seemingly - but not really random. The randomness depends on the seed used to initialzie the generator}
}

\newglossaryentry{mock}
{
    name=mock,
    description={A program object that pretends to behave like another one but has bare-bones implementation. Used for testing}
}